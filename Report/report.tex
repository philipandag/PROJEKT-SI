\documentclass{article}
\usepackage[T1]{fontenc}
\usepackage{lmodern}
\usepackage[polish]{babel}
\usepackage{graphicx}
\usepackage{float}
\usepackage{hyperref}

\usepackage[a4paper, margin=2.54cm]{geometry}

\title{Sprawozdanie z projektu ze Sztucznej Inteligencji}
\author{Filip Gołaś s188597 \\ Damian Jankowski s188597 \\ Mikołaj Storoniak s188806}

\begin{document}

\maketitle

\section{Wstęp}
Tutaj będzie bardzo ładny wstęp.

\section{Opis problemu}

Tutaj będzie bardzo ładny opis problemu.

\section{Opis modeli}

Tutaj z kolei będzie bardzo ładny opis modeli.

\section{Wyniki}

Natomiast tutaj będą bardzo ładne wyniki.

\section{Wnioski}

Najprawdopodobniej albo... albo szyny, 
które nie były... nie są równe, 
albo po prostu no, tak jak mówiłam we 
wcześniejszym wejściu, ee, yy, szyny... 
szyny były złe... a podwozie... podwozie... 
podwozie... podwozie też było złe. \cite{szyny}

\renewcommand{\refname}{Źródła}
\begin{thebibliography}{100}
    \bibitem{szyny} Wikipedia, 
    \textit{Szyny były złe} 
    \\\url{https://pl.wikipedia.org/wiki/Szyny_by%C5%82y_z%C5%82e}
    \bibitem{bib:2} Super strona,
    \textit{Jest super}
    \\\url{https://www.example.com}
    \bibitem{alicjabogdan} Marek Kubale,
    \textit{Łagodne wprowadzenie do algorytmów}, 2022
\end{thebibliography}

\end{document}


